\chapter*{Discussão: partidos, prefeitos, eleições e política orçamentária no Brasil}
\label{cap:conclusao}

Nos três esfoços de pesquisa que compõem esta tese, investiguei se há efeitos do partido governar o município sobre aspectos da competição eleitoral, da atividade legislativa e da vida partidária. O objetivo principal da tese, trabalhado sobretudo nos Capítulos~\ref{cap:eleicoes} e~\ref{cap:emendas}, doi observar a existência de vínculos partidário entre prefeitos e políticos situados nos demais níveis de governo e testar se a cooperação entre eles tem efeitos na performance eleitoral do partido e nas decisões orçamentárias no Congresso Nacional.

A narrativa sustentada pela tese é a de que prefeitos são atores políticos locais de destaque e dos quais candidatos nas eleições proporcionais dependem para construírem suas carreiras políticas. A importância eleitoral dos prefeitos teria como consequência o direcionamento da expressão particularista no Congresso Nacional e, eventualmente, nas Assembléias Legislativas, para a satisfação dos interesses de atores dentro do próprio partido e não necessariamente das demandas de bases eleitorais, noção difusa e imprecisa adotada como pressuposto em diversos trabalhos sobre os partidos brasileiros.

As evidências empíricas coletadas apontam que o fato do partido governar um município -- ou de vencer as eleições municipais -- afeta o desempenho eleitoral dos candidatos do partido nas eleições proporcionais e o total de recursos destinados à prefeitura apresentados por parlamentares de sua bancada na Câmara dos Deputados. Os resultados encontrados variam de acordo a condições eleitorais locais, no tempo e de acordo com as características do município, mas tendem a apontar sempre em uma mesma direção. Ainda que isoladamente essas duas evidências não sejam suficientes para corroborar a narrativa apresentada, em conjunto fazem com que tal narrativa seja uma interpretação bastante plausível sobre as relações partidárias entre políticos locais e representantes nacionais e estaduais em comparação às demais explicações disponíveis.

Na busca por evidências empíricas, a pesquisa desta tese produziu também um conjunto de resultados nulos. Não se encontrou evidências de que a vitória do partido nas eleições municipais afete o total de recursos destinados por parlamentares do partido a ESFLs situadas no município, no número de filiados à organização, nas receitas de campanhas futuras da sigla ou no tamanho das coligações eleitorais encabeçadas pelo partido. Também se produziu um conjunto de evidências cuja interpretação precisa ser mais bem pensada. Há efeitos, ainda que frágeis ou instáveis, do partido governar o município na performance eleitoral de seus candidatos a senador, govenador e presidente. Estes efeitos estão, por vezes, restritos a um partido, a um tipo de competidor ou a algumas eleições. Como não apontam para uma direção única ou têm aderência às explicações disponíveis, é melhor apontar para a necessidade de investigá-los com mais cuidado do que concluir algo a respeito deles.