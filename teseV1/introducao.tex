\chapter*{Introdução}
\label{cap:introducao}

O papel dos partidos políticos após a transição para a democracia é um dos temas mais discutidos na ciência política brasileira. Sobretudo no início da década de 90, a existência de um grande número de organizações partidárias foi apontada tanto como um sintoma de escolhas insituticionais equivocadas tomadas pelos atores políticos no passado, quanto como uma das causas do mal funcionamento institucional do sistema político político nacional. Além de numerosos, os partidos brasileiros seriam organizações incapazes de produzir ``bons governos'', conferir estabilidade e confiabilidade a políticas públicas e garantir o funcionamento pleno das instituições democráticas.

Entretanto, os avanços da democracia no Brasil nas últimas décadas também foram acompanhados pelo desenvolvimento de análises cada vez mais cuidadosas por parte ciência política brasileira. As teses mais pessimistas sobre os partidos políticos brasileiros -- sobretudo sobre sua atuação no Congresso Nacional -- foram bastante questionadas nas últimas duas décadas e diversos trabalhos contribuiram para melhorar compreensão de como tais organizações estão estruturadas e qual o seu papel para o desenvolvimento recente da democracia brasileira. Sob a influencia desta literatura, a presente tese tem como propósito contribuir para elucidar algumas questões ainda não investigadas sobre os partidos brasileiros. Especificamente, a pesquisa conduzida neste trabalho focaliza as relações partidárias entre governantes locais e políticos estaduais e nacionais e algumas se suas consequências nas arenas eleitoral e legislativa.

A tese reúne três trabalhos que poderiam ser lidos de maneira independente, mas que, seja pelos temas de que tratam, seja pela estratégias de investigação empírica, estão totalmente inter-relacionados. No primeiro trabalho analiso como os laços partidários entre prefeitos e candidatos a cargos eletivos nacionais e estaduais afetam o resultado das eleições. Em particular, o objetivo na primeira etapa da tese é observar se os partidos políticos têm desempenhos eleitorais superiores em municípios nos quais governam, sendo esta uma importante evidência da capacidade de membros de uma organização partidária cooperarem. Dedico especial atenção aos vínculos entre governantes municipais e deputados federais e estaduais, uma vez que relação entre parlamentares e suas bases eleitorais constitui um dos focos do debate acadêmico brasileiro.

Se no primeiro trabalho procuro evidências de que há cooperação entre prefeitos e candidatos de um mesmo partido durante eleições, no segundo tento compreender como os vínculos entre prefeitos e deputados federais influenciam a atuação dos partidos no Congresso Nacional. Parlamentares na Câmara dos Deputado decidem ao longo do mandato sobre como alocar recursos que beneficiam prefeituras e organizações diretamente associadas com a vida política local. Tradicionalmente, parte da literatura sobre o Congresso Nacional investigou o uso estratégico de recursos orçamentárias federais (sobretudo das emendas individuais ao orçamento) para construção de carreiras parlamentares e reputação pessoal sem, contudo, dar a devida atenção às relações partidárias que permeiam o processo decisório. A centralidade da noção de \emph{conexão eleitoral} ofuscou a importância das conexões partidárias que influenciam diretamente as decisões dos deputados federais. Assim, o objetivo nesta segunda etapa da tese é investigar como os vínculos entre governantes locais e parlamentares na Câmara dos Deputados afeta diretamente alocação de recursos federais por meio das emendas individuais ao orçamento.

Tomados em conjunto, esses dois primeiros trabalhos constroem uma narrativa sobre o funcionamento das relações entre políticos de um partido situados em diferentes níveis de governo. Prefeitos são atores políticos privilegiados durante as eleições nacionais e estaduais pela sua capacidade de atrair votos para candidatos de seu partido. Deputados federais -- e quiçá deputados estaduais -- tomam decisões de fundamental importância para governantes locais. Os partidos políticos estruturam as relações entre políticos locais e nacionais/estaduais e tornam a cooperação viável. Os resultados encontrados nas duas primeiras etapas da pesquisa dão suporte a esta narrativa.

O terceiro esforço investigavito que integra a tese, bastante mais modesto e menos bem sucedido que os dois primeiros, tenta completar a narrativa a partir do exame de uma pergunta pouco explorada na literatura: como partidos vitoriosos conquistam vantagens eleitorais? Se prefeitos são capazes de influenciar positivamente a performance eleitoral de candidatos de sua organização partidária, quais são os mecanismos pelos quais o fazem? Nesta etapa do trabalho procuro compreender se partidos vitoriosos atraem mais apoios para suas candidaturas futuras, seja recrutando novos membros, conquistando mais financiadores de campanha ou construindo coligações eleitorais maiores. Diferentemente das etapas anteriores da pesquisa, os resultados encontrados nesta última parte da tese não sustentam as suspeitas iniciais.

Há diversas observações importantes para se fazer sobre a confecção desta tese antes de avançar.

Apesar das temáticas dos trabalhos produzidos para esta tese estarem conectados, o fio condutor da tese é o método. Mais especificamente, o desenho de pesquisa apresentado no Capítulo~\ref{cap:eleicoes} se repete nos demais. Em todos os trabalhos o problema pode ser traduzido como a investiação do impacto do partido vencer as eleições para prefeito -- ou ainda, o impacto do partido governar o município -- sobre um conjunto de variáveis: o desempenho eleitoral dos candidatos do partido nas próximas eleições; as decisões da bancada do partido na Câmara dos Deputados sobre o orçamento; ou o apoio político e financeiro para o partido em campanhas futuras. Para estimar adequadamente o impacto da vitória no município sobre tais variáveis, utilizo um desenho de regressão descontínua, estratégia que se tornou bastante popular na ciência política e na economia nos últimos anos.

Mesmo adotando uma estratégia empírica que permite recuperar a noção \emph{impacto causal} na análise dos resultados, o objetivo da tese deve ser pensado como exploratório por pelo menos duas razões.

Em primeiro lugar, muitos dos resultados encontrados ainda são bastante instáveis e merecem exame mais cuidadoso no futuro. Há consequências diretas desta aspecto da pesquisa na forma do texto e na apresentação dos resultados. Em vez de me concentrar nas escolhas mais econômicas, fortuitas e definitivas dos resultados, como haveria de ser em uma publicação dos mesmos trabalhos em revistas acadêmicas, decidi apresentar todos os procedimentos adotados, variações de medidas e de estimativas, resultados nulos e conjecturas que me pareceram necessárias para refletir sobre os problemas investigados. A expectativa é facilitar a elaboração de sugestões e críticas sobre a análise empírica para poder avançar.

Em segundo lugar, as evidências empíricas produzidas nesta tese dão suporte à narrativa apresentada acima, mas a tese carece de uma reflexão mais ampla sobre as consequências teóricas dos resultados encontrados. Há um desbalanceamento proposital em favor de um texto voltado ao teste de hipóteses e à variação de aspectos analíticos em detrimento do desenvolvimento teórico dos problemas 00.

Para viabilizar a leitura e interpretação dos resultados produzidos para a tese, optei por não apresentar coeficientes de regressão em tabelas, exceto no último capítulo. Praticamente todos os resultados são apresentados graficamente e, normalmente, para todas as formas de estimar e medidas adotadas. Por parcimônia, muitos dos resultados e variações dos gráficos são deixados para o Anexo, que não integra este texto, mas pode ser acessado junto com a documentação e outros materiais da tese. Além da opção pela apresentação gráfica dos resultados, também dou preferência à utilização das estatísticas mais simples que os problemas, por vezes bastante complexos, permitiram.

O texto apresentado à Comissão Avaliadora não contém o Anexo (assim referido ao longo da tese), documentação, scripts e demais materiais utilizados na produção da tese. Todos estes materiais, bem como as versões do texto, podem ser encontrados em um repositório online criado exclusivamente para esta tese no endereço: https://github.com/leobarone/tese. Os scripts de construção de dados, gráficos e tabelas foram escritos em liguagem \emph{R}, utilizada para a construção das análises. Os documentos usados na elaboração do texto foram escritos em \LaTeX e também estão disponíveis no repositório online.

A tese está dividida em quatro capítulos, além desta Introdução e de uma breve seção de Discussão ao final. Cada um dos três primeiros capítulos correspondem a uma etapa da pesquisa. O Capítulo~\ref{cap:financiamento}, que originalmente não integrava o plano da tese, apenas retoma os demais e reve as análises empíricas com a finalidade de validar os resultados obtidos, como será explicado adiante.