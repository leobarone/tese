\documentclass[12pt,twoside,a4paper]{book}

% Enconding
\usepackage[brazilian]{babel}
% Enconding
\usepackage[utf8]{inputenc}
% Pdf e Jpeg como figuras
\usepackage[T1]{fontenc}
% Lingua
\usepackage[pdftex]{graphicx}
% Espaçamento flexível
\usepackage{setspace}
% Identação do primeiro parágrafo
\usepackage{indentfirst}
% Índice remissivo
\usepackage{makeidx}
% Bibliografia/Indice/Conteudo na Table of Contents
\usepackage[nottoc]{tocbibind}         
% Adobe Courier no lugar de Computer Modern Typewriter
\usepackage{courier}

% Bibliografia
\usepackage[portuguese]{babelbib}
\usepackage{natbib} 
% Margens
\usepackage[a4paper,top=3.5cm,bottom=2.5cm,left=3.5cm,right=2.5cm]{geometry} % margens

\usepackage{rotating}


% Outros
\usepackage{type1cm}
\usepackage{listings}
\usepackage{titletoc}
\usepackage[font=small,format=plain,labelfont=bf,up,textfont=it,up]{caption}
\usepackage[usenames,svgnames,dvipsnames]{xcolor}
\fontsize{60}{62}\usefont{OT1}{cmr}{m}{n}{\selectfont}

\usepackage{fancyhdr}
\pagestyle{fancy}
\fancyhf{}
\renewcommand{\chaptermark}[1]{\markboth{\MakeUppercase{#1}}{}}
\renewcommand{\sectionmark}[1]{\markright{\MakeUppercase{#1}}{}}
\renewcommand{\headrulewidth}{0pt} 
% Caminho dos arquivos de figura
\graphicspath{{./figuras/}}


\frenchspacing                          % arruma o espaço: id est (i.e.) e exempli gratia (e.g.) 

% Índice remissivo
\makeindex
\raggedbottom                           % para não permitir espaços extra no texto
\fontsize{60}{62}\usefont{OT1}{cmr}{m}{n}{\selectfont}
\cleardoublepage
\normalsize

\lstset{ %
language=Java,                  % choose the language of the code
basicstyle=\footnotesize,       % the size of the fonts that are used for the code
numbers=left,                   % where to put the line-numbers
numberstyle=\footnotesize,      % the size of the fonts that are used for the line-numbers
stepnumber=1,                   % the step between two line-numbers. If it's 1 each line will be numbered
numbersep=5pt,                  % how far the line-numbers are from the code
showspaces=false,               % show spaces adding particular underscores
showstringspaces=false,         % underline spaces within strings
showtabs=false,                 % show tabs within strings adding particular underscores
frame=single,	                % adds a frame around the code
framerule=0.6pt,
tabsize=2,	                    % sets default tabsize to 2 spaces
captionpos=b,                   % sets the caption-position to bottom
breaklines=true,                % sets automatic line breaking
breakatwhitespace=false,        % sets if automatic breaks should only happen at whitespace
escapeinside={\%*}{*)},         % if you want to add a comment within your code
backgroundcolor=\color[rgb]{1.0,1.0,1.0}, % choose the background color.
rulecolor=\color[rgb]{0.8,0.8,0.8},
extendedchars=true,
xleftmargin=10pt,
xrightmargin=10pt,
framexleftmargin=10pt,
framexrightmargin=10pt
}


\begin{document}

\frontmatter 
% cabeçalho para as páginas das seções anteriores ao capítulo 1 (frontmatter)
\fancyhead[RO]{{\footnotesize\rightmark}\hspace{2em}\thepage}
\setcounter{tocdepth}{2}
\fancyhead[LE]{\thepage\hspace{2em}\footnotesize{\leftmark}}
\fancyhead[RE,LO]{}
\fancyhead[RO]{{\footnotesize\rightmark}\hspace{2em}\thepage}

\onehalfspacing  % espaçamento

% CAPA
\thispagestyle{empty}
\begin{center}
% Distância do titulo ao topo
    \vspace*{2.3cm}
% Titulo
    \textbf{\Large{Eleições, Partidos e Política Orçamentária no Brasil: explorando os efeitos das eleições locais na política nacional
}}\\
% Distância do texto acima
    \vspace*{1.8cm}
% Nome do autor
    \Large{Leonardo Sangali Barone}
% Distância do texto acima    
    \vskip 2cm
    \textsc{
    Tese apresentada à\\[-0.25cm] 
    Escola de Administração de Empresas\\[-0.25cm]
    de São Paulo da\\[-0.25cm]
	Fundação Getulio Vargas\\[-0.25cm]
    para obtenção do título de\\[-0.25cm]
    Doutor em Administração Pública e Governo}
% Distância do texto acima    
    \vskip 2cm
    Programa: Doutorado em Administração Pública e Governo\\
    Orientador: Prof. Dr. Geroge Avelino Filho\\
% Distância do texto acima
    \vskip 4cm
% Distância do texto acima
    \normalsize{São Paulo\\Fevereiro de 2014}
\end{center}

% Página de rosto
\newpage
\thispagestyle{empty}
    \begin{center}
        \vspace*{2.3 cm}
        \textbf{\Large{Eleições, Partidos e Política Orçamentária no Brasil: explorando os efeitos das eleições locais na política nacional}}\\
        \vspace*{2 cm}
    \end{center}

    \vskip 2cm

    \begin{flushright}
	Esta é a versão original da tese elaborada pelo\\
	candidato Leonardo Sangali Barone, tal como \\
	submetida à Comissão Avaliadora.
    \end{flushright}

\pagebreak
\newpage
\thispagestyle{empty}
    \begin{center}
        \vspace*{2.3 cm}
        \textbf{\Large{Eleições, Partidos e Política Orçamentária no Brasil: explorando os efeitos das eleições locais na política nacional}}\\
        \vspace*{2 cm}
    \end{center}

    \vskip 2cm

    \begin{flushright}
    Esta é a versão original da tese elaborada pelo\\
    candidato Leonardo Sangali Barone, tal como \\
    submetida à Comissão Avaliadora.

    \vskip 2cm

    \end{flushright}
    \vskip 4.2cm

    \begin{quote}
    \noindent Comissão Avaliadora:
    
    \begin{itemize}
		\item Prof. Dr. George Avelino Filho (orientador) - FGV-EAESP
		\item Prof. Dr. Ciro Biderman - FGV-EAESP
		\item Prof. Dr. Fernando Luiz Abrucio - FGV-EAESP
        \item Prof. Dr. Sergio Pinheiro Firpo - FGV-EESP
        \item Prof. Dr. Fernando de Magalhães Papaterra Limongi - FFLCH-USP
    \end{itemize}
      
    \end{quote}

\pagebreak

\newpage
\thispagestyle{empty}

    \vspace*{17 cm}

    \begin{flushright}
    Para Aninha
    \end{flushright}

\pagebreak

% Numeração
\pagenumbering{roman}

% Insere Agradecimentos
\input agradecimentos
% Insere Resumo Portugues
\input resumoPortugues
% Insere Resumo Ingles
\input resumoIngles

% Imprime o sumário
\tableofcontents

% Lista de Abreviaturas
\chapter{Lista de Abreviaturas}
\begin{tabular}{ll}
         DEM         & Democratas\\
         ESFL         & Entidade sem fins lucrativos\\
         IBGE           & Instituto Brasileiro de Geografia e Estatística\\
         IPEA         & Instituto de Pesquisa Econômica Aplicada\\
         IPEADATA	  & Repositório de dados do IPEA\\
         LDO			& Lei de Diretrizes Orçamentárias\\
         LOA			& Lei Orçamentária Anual\\
         PDT         & Partido Democrático Trabalhista\\
         PL			& Partido Liberal\\
         PLOA		& Projeto de Lei Orçamentária Anual\\
         PMDB         & Partido do Movimento Democrático Brasileiro\\
         PP         & Partido Progressista\\
         PPA			& Plano Plurianual\\
         PPB			& Partido Progressista Brasileiro\\
         PR         & Partido da República\\
         PRONA		& Partido da Reedificação da Ordem Nacional\\
         PSD			& Partido Social Democrático\\
         PSB         & Partido Socialista Brasileiro\\
         PSDB         & Partido da Social Democracia Brasileira\\
         PT         & Partido dos Trabalhadores\\
         RDD         & Desenho de Regressão Descontínua\\
         SIGA		& Portal SIGA Brasil - Senado Federal\\
         STN			& Secretaria do Tesouro Nacional\\
         TSE         & Tribunal Superior Eleitoral\\


\end{tabular}

% Lista de Símbolos
\chapter{Lista de Símbolos}
\begin{tabular}{ll}
        $\rho$    & Efeito causal do tratamento\\
        $\tau$    & Margem de Vitória\\
        $\Delta$      & Módulo da Margem de Vitória\\
\end{tabular}

% Lista de Figuras
\listoffigures
% Lista de Tabelas        
\listoftables
       
\mainmatter

% cabeçalho para as páginas de todos os capítulos
%\fancyhead[RE,LO]{\thesection}

% Espaçamento (simples ou 1,5)
%\singlespacing
\onehalfspacing

% Introdução
\input introducao
% Capítulo 1
\input cap1
% Capítulo 2
\input cap2
% Capítulo 3
\input cap3
% Capítulo 4
\input cap4
% Conclusão
\input conclusao


% Referências Bibliográficas
\backmatter \singlespacing
% Estilo das referências
\bibliographystyle{plainnat-ime} 
\bibliography{bibliografia}

\renewcommand{\chaptermark}[1]{\markboth{\MakeUppercase{\appendixname\ \thechapter}} {\MakeUppercase{#1}} }
\fancyhead[RE,LO]{}


\end{document}