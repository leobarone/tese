\chapter*{Resumo}

\noindent BARONE, L. S. \textbf{Eleições, Partidos e Política Orçamentária no Brasil: explorando os efeitos das eleições locais na política nacional}. 
2010. %120 f.
Tese (Doutorado) - Escola de Administração de Empresas de São Paulo,
Fundação Getulio Vargas, São Paulo, 2014.
\\

Esta tese investiga como as relações partidárias entre prefeitos brasileiros e políticos nos níveis do mesmo partido explica a performance eleitoral do partido e as deciões dos partidos no Congresso. Em particular, o objetivo da pesquisa tese é testar se o fato do partido governar um município afeta o desempenho de seus candidatos nas eleições nacionais e estaduais, as decisões de seus parlamentares no processo orçamentário e a capacidade do partido em obter apoios políticos e financeiros para suas campanhas futuras. Os efeitos do partido governar um município são estimados utilizando um desenho de regressão descontínua e para as eleições para prefeito de 1996 a 2008.
\newline
\\

\noindent \textbf{Keywords:} Partidos políticos, prefeitos, eleições nacionais, eleições locais, emendas individuais, desenho de regressão descontínua.

