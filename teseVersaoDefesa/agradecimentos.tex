
\chapter*{Agradecimentos (preliminares)}

Antes de concluir esta tese é necessário agradecer (e, eventualmente, pedir desculpas) a muitas pessoas. Deixarei para escrever agradecimentos mais completos (e carinhosos) na versão final deste texto. São muitos e escrevê-los requer tempo. Por ora, faço apenas os agradecimentos institucionais necessários e os agradecimentos às pessoas que ainda terão tarefas em virtude da elaboração desta tese.

Agradeço, assim, à FGV-EAESP pelas oportunidades nestes 12 anos de vida universitária e acadêmica. À CAPES, pela isenção das mensalidades iniciais do curso e pela bolsa de doutorado sanduíche. Ao IPEA, em especial, e também ao CEPESP, CEBRAP e CEM, que foram, na prática, as ``agências de fomento'' que me deram a oportunidade de seguir exclusivamente como pesquisador ao longo do programa de doutorado. À University of California, San Diego, que me acolheu e contribuiu para o desenvolvimento deste trabalho.

Enormes agradecimentos ao meu orientador, Prof. George Avelino, e co-orientadores, Prof. Ciro Biderman e Prof. Scott Desposato, pela paciência, parceria e confiança, e aos demais membros da banca avaliadora, Prof. Fernando Abrucio, Prof. Fernando Limongi e Prof Sergio Firpo, não apenas pela paciência de ler este trabalho, mas por terem me inflenciado diretamente em sala de aula.